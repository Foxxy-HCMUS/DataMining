\documentclass[a4paper, 12pt]{article}

\usepackage{multirow}
\usepackage[table,xcdraw]{xcolor}
\usepackage{enumerate}
\usepackage{graphicx}
\usepackage[T5]{fontenc}
\usepackage[utf8]{inputenc}
\usepackage[margin = 2cm]{geometry}
\usepackage{amsfonts, amsmath, amssymb}
\usepackage[none]{hyphenat}
\usepackage{fancyhdr}
\usepackage{float}
\usepackage{hyperref}
\usepackage{caption}
\usepackage[nottoc, notlot, notlof]{tocbibind}

% \captionsetup[table]{skip=5pt}
\pagestyle{fancy}
\fancyhead[L]{Trường Đại học Khoa học Tự nhiên - ĐHQG TP.HCM}
\fancyhead[R]{Nhóm Just $4^{th}$}

\begin{document}

\begin{titlepage}
    \begin{center}
        \vspace*{1cm}
        \Large\textbf{Đại học Quốc gia TP.HCM\\Trường Đại học Khoa học Tự nhiên}\\

        \vfill
        \line(1,0){450}\\[4mm]
        \LARGE\textbf{\MakeUppercase{Classification \& Clustering}}\\[3mm]
        \Large{Khai thác dữ liệu \& Ứng dụng}\\[3mm]
        \Large{Nguyễn Bảo Long - MSSV: 18120201}\\
        \Large{Huỳnh Long Nam - MSSV: 18120212}
        \line(1,0){430}\\
        \vfill

        \vfill
        TP Hồ Chí Minh, ngày 02/12/2020
    \end{center}
\end{titlepage}

\tableofcontents
\thispagestyle{empty}
\clearpage

\section{Thông tin chung}

\begin{enumerate}
    \item Link GitHub: \url{https://github.com/baolongnguyenmac/DataMining-Lab3}
    \item Thông tin thành viên nhóm
    \begin{table}[H]
        \begin{center}
            \begin{tabular}{|c|c|c|c|}
            \hline
            STT & Họ tên          & MSSV     & Email                         \\ \hline
            1   & Nguyễn Bảo Long & 18120201 & 18120201@student.hcmus.edu.vn \\ \hline
            2   & Huỳnh Nam Long  & 18120212 & 18120212@student.hcmus.edu.vn         \\ \hline
            \end{tabular}
            \caption{Bảng thông tin thành viên nhóm}
        \end{center}
    \end{table}

    \item Tỷ lệ tham gia công việc
    \begin{table}[H]
        \begin{center}
            \begin{tabular}{|c|c|l|c|}
            \hline
            STT & Họ tên                    & \multicolumn{1}{c|}{Công việc}     & Tỷ lệ hoàn thành      \\ \hline
            1   & \multirow{4}{*}{18120201} & Tiền xử lý dữ liệu                 & \multirow{4}{*}{50\%} \\ \cline{1-1} \cline{3-3}
            2 &  & Phân lớp dữ liệu bằng Weka Explorer      &  \\ \cline{1-1} \cline{3-3}
            3 &  & Cài đặt và kiểm thử thuật toán K-Means   &  \\ \cline{1-1} \cline{3-3}
            4 &  & Trình bày báo cáo                        &  \\ \hline
            5   & \multirow{3}{*}{18120212} & Phân lớp dữ liệu bằng Experimenter & \multirow{3}{*}{50\%} \\ \cline{1-1} \cline{3-3}
            6 &  & Đánh giá phương pháp phân lớp            &  \\ \cline{1-1} \cline{3-3}
            7 &  & Cài đặt và kiểm thử thuật toán K-Medoids &  \\ \hline
            \end{tabular}
            \caption{Bảng phân chia công việc}
        \end{center}
    \end{table}
\end{enumerate}
\clearpage

\section{Ý tưởng tiền xử lý dữ liệu}

\begin{itemize}
    \item Chi tiết quá trình tiền xử lý dữ liệu được trình bày trong file `Preprocess.ipynb'
    \item Ý tưởng chung
    \begin{itemize}
        \item Xoá các thuộc tính có tỷ lệ thiếu dữ liệu lớn hơn hoặc bằng 50\%
        \item Xoá các dữ liệu dạng IDentification
        \item Điền giá trị thiếu tại các cột có kiểu dữ liệu dạng số bằng giá trị trung bình 
        \item Điều giá trị thiếu cho các cột có kiểu dữ liệu định danh bằng giá trị mode
    \end{itemize}
\end{itemize}
\clearpage

\section{Đánh giá phương pháp phân lớp}

\begin{itemize}
    \item Phương pháp phân lớp nào thường cho kết quả cao nhất?\\
    $\rightarrow$

    \item Phương pháp nào không thực hiện tốt và tại sao?\\
    $\rightarrow$

    \item Tại sao ta sử dụng phiên bản đã rời rạc hóa của tập dữ liệu nếu tập dữ liệu đã được rời rạc hóa?\\
    $\rightarrow$
    \item Việc rời rạc hóa và cách rời rạc hóa có ảnh hưởng đến kết quả phân lớp hay không, nếu có thì ảnh hưởng thể nào?\\
    $\rightarrow$

    \item Chiến lược nào trong ba chiến lược đánh giá đã đánh giá quá cao (overestimate) độ chính xác và tại sao?\\
    $\rightarrow$

    \item Chiến lược nào đánh giá thấp (underestimate) độ chính xác và tại sao?\\
    $\rightarrow$
\end{itemize}
\clearpage

\end{document}